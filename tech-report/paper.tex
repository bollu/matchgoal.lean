\def\paperversiondraft{draft}
\def\paperversionblind{normal}
\def\paperversioncameraACM{cameraACM}

% If no draft paper-version is requested, compile in 'normal' mode
\ifx\paperversion\paperversiondraft
\else
  \ifx\paperversion\paperversioncameraACM
  \else
  \def\paperversion{normal}
  \fi
\fi

\ifx\paperversion\paperversioncameraACM
  \documentclass[sigplan]{acmart}
\else
  \documentclass[review, anonymous, sigplan]{acmart}
\fi

\def\acmversionanonymous{anonymous}
\def\acmversionjournal{journal}
\def\acmversionnone{none}

\makeatletter
\if@ACM@anonymous
  \def\acmversion{anonymous}
\else
  \def\acmversion{journal}
\fi
\makeatother


% https://users.cs.northwestern.edu/~jesse/code/latex/ottalt/Makefile
% \usepackage{ottalt}
\usepackage{colortbl}

% 'draftonly' environment
\usepackage{environ}
\ifx\paperversion\paperversiondraft
\newenvironment{draftonly}{}{}
\else
\NewEnviron{draftonly}{}
\fi

% Most PL conferences are edited by conference-publishing.com. Follow their
% advice to add the following packages.
%
% The first enables the use of UTF-8 as character encoding, which is the
% standard nowadays. The second ensures the use of font encodings that support
% accented characters etc. (Why should I use this?). The mictotype package
% enables certain features 'to­wards ty­po­graph­i­cal per­fec­tion
\usepackage[utf8]{inputenc}
\usepackage[T1]{fontenc}
\usepackage{microtype}

\usepackage{minted}
\usemintedstyle{colorful}
% \newminted[mlir]{tools/MLIRLexer.py:MLIRLexerOnlyOps -x}{mathescape}
% \newminted[xdsl]{tools/MLIRLexer.py:MLIRLexer -x}{mathescape, style=murphy}

% \usepackage{amsmath}
% \usepackage{amssymb}
\usepackage{xargs}
\usepackage{lipsum}
\usepackage{xparse}
\usepackage{xifthen, xstring}
\usepackage{xspace}
\usepackage{marginnote}
\usepackage{etoolbox}
\usepackage{tikz}
\usepackage{quiver}

%%%%%%%%%%%%%%%%%%%%%%%%%%%%%%%%%%%%%%%%%%%%%%%%%%%%%%%%%%%%%%%%%%%%%%%%%%%%%%%
% Broaden margins to make room for todo notes
%%%%%%%%%%%%%%%%%%%%%%%%%%%%%%%%%%%%%%%%%%%%%%%%%%%%%%%%%%%%%%%%%%%%%%%%%%%%%%%

\makeatletter
\patchcmd{\@addmarginpar}{\ifodd\c@page}{\ifodd\c@page\@tempcnta\m@ne}{}{}
\makeatother
\ifx\paperversion\paperversiondraft
  \makeatletter
  \if@ACM@journal
    \geometry{asymmetric}
    \paperwidth=\dimexpr \paperwidth + 3.5cm\relax
    \oddsidemargin=\dimexpr\oddsidemargin + 0cm\relax
    \evensidemargin=\dimexpr\evensidemargin + 0cm\relax
    \marginparwidth=\dimexpr \marginparwidth + 3cm\relax
    \setlength{\marginparwidth}{4.6cm}
    % This makeatletter box helps to move notes to the right
    \makeatletter
    \long\def\@mn@@@marginnote[#1]#2[#3]{%
      \begingroup
        \ifmmode\mn@strut\let\@tempa\mn@vadjust\else
          \if@inlabel\leavevmode\fi
          \ifhmode\mn@strut\let\@tempa\mn@vadjust\else\let\@tempa\mn@vlap\fi
        \fi
        \@tempa{%
          \vbox to\z@{%
            \vss
            \@mn@margintest
            \if@reversemargin\if@tempswa
                \@tempswafalse
              \else
                \@tempswatrue
            \fi\fi
            %\if@tempswa
              \rlap{%
                \if@mn@verbose
                  \PackageInfo{marginnote}{xpos seems to be \@mn@currxpos}%
                \fi
                \begingroup
                  \ifx\@mn@currxpos\relax\else\ifx\@mn@currxpos\@empty\else
                      \kern-\dimexpr\@mn@currxpos\relax
                  \fi\fi
                  \ifx\@mn@currpage\relax
                    \let\@mn@currpage\@ne
                  \fi
                  \if@twoside\ifodd\@mn@currpage\relax
                      \kern\oddsidemargin
                    \else
                      \kern\evensidemargin
                    \fi
                  \else
                    \kern\oddsidemargin
                  \fi
                  \kern 1in
                \endgroup
                \kern\marginnotetextwidth\kern\marginparsep
                \vbox to\z@{\kern\marginnotevadjust\kern #3
                  \vbox to\z@{%
                    \hsize\marginparwidth
                    \linewidth\hsize
                    \kern-\parskip
                    \marginfont\raggedrightmarginnote\strut\hspace{\z@}%
                    \ignorespaces#2\endgraf
                    \vss}%
                  \vss}%
              }%
          }%
        }%
      \endgroup
    }
    \makeatother
  \else
    \paperwidth=\dimexpr \paperwidth + 6cm\relax
    \oddsidemargin=\dimexpr\oddsidemargin + 3cm\relax
    \evensidemargin=\dimexpr\evensidemargin + 3cm\relax
    \marginparwidth=\dimexpr \marginparwidth + 3cm\relax
    \setlength{\marginparwidth}{4.6cm}
  \fi
  \makeatother
\fi

%%%%%%%%%%%%%%%%%%%%%%%%%%%%%%%%%%%%%%%%%%%%%%%%%%%%%%%%%%%%%%%%%%%%%%%%%%%%%%%
% Add \createtodoauthor command
%%%%%%%%%%%%%%%%%%%%%%%%%%%%%%%%%%%%%%%%%%%%%%%%%%%%%%%%%%%%%%%%%%%%%%%%%%%%%%%

\usepackage[textsize=tiny]{todonotes}
\usepackage[normalem]{ulem}

\makeatletter
\font\uwavefont=lasyb10 scaled 652
\newcommand\colorwave[1][blue]{\bgroup\markoverwith{\lower3\p@\hbox{\uwavefont\textcolor{#1}{\char58}}}\ULon}

\newcommand\highlight[2]{{\color{#1}{\colorwave[#1]{#2}}}}
\makeatother

\makeatletter
\newcommand\InFloat[2]{\ifnum\@floatpenalty<0\relax#1\else#2\fi}
\makeatother

\ifx\paperversion\paperversiondraft
\newcommand\createtodoauthor[2]{
  \def\tmpdefault{emptystring}
  \expandafter\newcommand\csname #1\endcsname[2][\tmpdefault]{% comment to avoid spurious whitespace
    \def\tmp{##1}% comment to avoid spurious whitespace
    \InFloat{
        \smash{
	  \marginnote{
	    \todo[inline,linecolor=#2,backgroundcolor=#2,bordercolor=#2]
	      {\textbf{#1 (Figure):} ##2}
          }
        }
    }{\ifthenelse{\equal{\tmp}{\tmpdefault}} % Is there text to highlight?
      {\todo[linecolor=#2,backgroundcolor=#2,bordercolor=#2]{\textbf{#1:} ##2}\ignorespaces}
      {\ifthenelse{\equal{##2}{}} % Is there a note?
        {\highlight{#2}{##1}}
        {\highlight{#2}{##1}\todo[linecolor=#2,backgroundcolor=#2,bordercolor=#2]
	  {\textbf{#1:} ##2}% comment to avoid spurious whitespace
	}% comment to avoid spurious whitespace
      }% comment to avoid spurious whitespace
    }% comment to avoid spurious whitespace
  }
}
\else
\newcommand\createtodoauthor[2]{%
\expandafter\newcommand\csname #1\endcsname[2][]{##1}%
}%
\fi

\ifx\acmversion\acmversionjournal
  %% Journal information (used by PACMPL format)
  %% Supplied to authors by publisher for camera-ready submission
  \acmJournal{PACMPL}
  \acmVolume{1}
  \acmNumber{1}
  \acmArticle{1}
  \acmYear{2017}
  \acmMonth{1}
  \acmDOI{10.1145/nnnnnnn.nnnnnnn}
  \startPage{1}
\else
  %% Conference information (used by SIGPLAN proceedings format)
  %% Supplied to authors by publisher for camera-ready submission
  \acmConference[PL'17]{ACM SIGPLAN Conference on Programming Languages}{January 01--03, 2017}{New York, NY, USA}
  \acmYear{2017}
  \acmISBN{978-x-xxxx-xxxx-x/YY/MM}
  \acmDOI{10.1145/nnnnnnn.nnnnnnn}
  \startPage{1}
\fi

%% Copyright information
%% Supplied to authors (based on authors' rights management selection;
%% see authors.acm.org) by publisher for camera-ready submission
%\setcopyright{none}             %% For review submission
%\setcopyright{acmcopyright}
%\setcopyright{acmlicensed}
%\setcopyright{rightsretained}
%\copyrightyear{2017}           %% If different from \acmYear


%% Bibliography style
\bibliographystyle{ACM-Reference-Format}
%% Citation style
%% Note: author/year citations are required for papers published as an
%% issue of PACMPL.
%\citestyle{acmauthoryear}  %% For author/year citations
%\citestyle{acmnumeric}     %% For numeric citations
%\setcitestyle{nosort}      %% With 'acmnumeric', to disable automatic
                            %% sorting of references within a single citation;
                            %% e.g., \cite{Smith99,Carpenter05,Baker12}
                            %% rendered as [14,5,2] rather than [2,5,14].
%\setcitesyle{nocompress}   %% With 'acmnumeric', to disable automatic
                            %% compression of sequential references within a
                            %% single citation;
                            %% e.g., \cite{Baker12,Baker14,Baker16}
                            %% rendered as [2,3,4] rather than [2-4].


% We use the following color scheme
%
% This scheme is both print-friendly and colorblind safe for
% up to four colors (including the red tones makes it not
% colorblind safe any more)
%
% https://colorbrewer2.org/#type=qualitative&scheme=Paired&n=4

\definecolor{pairedNegOneLightGray}{HTML}{cacaca}
\definecolor{pairedNegTwoDarkGray}{HTML}{827b7b}
\definecolor{pairedOneLightBlue}{HTML}{a6cee3}
\definecolor{pairedTwoDarkBlue}{HTML}{1f78b4}
\definecolor{pairedThreeLightGreen}{HTML}{b2df8a}
\definecolor{pairedFourDarkGreen}{HTML}{33a02c}
\definecolor{pairedFiveLightRed}{HTML}{fb9a99}
\definecolor{pairedSixDarkRed}{HTML}{e31a1c}

\createtodoauthor{grosser}{pairedOneLightBlue}
\createtodoauthor{bollu}{pairedTwoDarkBlue}
\createtodoauthor{authorThree}{pairedThreeLightGreen}
\createtodoauthor{authorFour}{pairedFourDarkGreen}
\createtodoauthor{authorFive}{pairedFiveLightRed}
\createtodoauthor{authorSix}{pairedSixDarkRed}

\graphicspath{{./images/}}

% Define macros that are used in this paper
%
% We require all macros to end with a delimiter (by default {}) to enusure
% that LaTeX adds whitespace correctly.
\makeatletter
\newcommand\requiredelimiter[2][########]{%
  \ifdefined#2%
    \def\@temp{\def#2#1}%
    \expandafter\@temp\expandafter{#2}%
  \else
    \@latex@error{\noexpand#2undefined}\@ehc
  \fi
}
\@onlypreamble\requiredelimiter
\makeatother

\newcommand\newdelimitedcommand[2]{
\expandafter\newcommand\csname #1\endcsname{#2}
\expandafter\requiredelimiter
\csname #1 \endcsname
}


\newcommand{\mlir}{\ensuremath{\texttt{MLIR}}}
\requiredelimiter{\mlir}

\newcommand{\mlirflat}{\ensuremath{\texttt{MLIR}^\flat}}
\requiredelimiter{\mlirflat}

\newcommand{\val}{\ensuremath{\texttt{Val}}}
\requiredelimiter{\val}

\newcommand{\stxlet}{\ensuremath{\texttt{let}}}
\requiredelimiter{\stxlet}

\newcommand{\stxMLIR}{\ensuremath{\texttt{stxMLIR}}}
\requiredelimiter{\stxMLIR}

\newcommand{\denoteMLIR}[1]{\ensuremath{\llbracket \texttt{MLIR}(#1) \rrbracket}}

\newcommand{\libMLIR}{\ensuremath{\texttt{libMLIR}}}
\requiredelimiter{\libMLIR}

\newcommand{\implMLIR}[1]{\ensuremath{\texttt{implMLIR(#1)}}}

\newdelimitedcommand{toolname}{Tool}

\usepackage{booktabs}
\newcommand{\ra}[1]{\renewcommand{\arraystretch}{#1}}

% \circled command to print a colored circle.
% \circled{1} pretty-prints "(1)"
% This is useful to refer to labels that are embedded within figures.
\usepackage{tikz}
\usetikzlibrary{arrows}
\usetikzlibrary{shapes}
\DeclareRobustCommand\circled[2][]{\ifthenelse{\isempty{#1}}{\tikz[baseline=(char.base)]{\node[shape=circle,fill=pairedOneLightBlue,inner sep=1pt] (char) {#2};}}{\autoref{#1}: \hyperref[#1]{\tikz[baseline=(char.base)]{\node[shape=circle,fill=pairedOneLightBlue,inner sep=1pt] (char) {#2};}}}}



% listings don't write "Listing" in autoref without this.
\providecommand*{\listingautorefname}{Listing}

\begin{document}

%% Title information
\title{\mlir{}: Its Syntax and Semantics}
                                      %% when present, will be used in
                                      %% header instead of Full Title.
% \subtitle{Subtitle}                   %% \subtitle is optional


%% Author information
%% Contents and number of authors suppressed with 'anonymous'.
%% Each author should be introduced by \author, followed by
%% \authornote (optional), \orcid (optional), \affiliation, and
%% \email.
%% An author may have multiple affiliations and/or emails; repeat the
%% appropriate command.
%% Many elements are not rendered, but should be provided for metadata
%% extraction tools.
\author{First1 Last1}
\authornote{with author1 note}          %% \authornote is optional;
                                      %% can be repeated if necessary
\orcid{nnnn-nnnn-nnnn-nnnn}             %% \orcid is optional
\affiliation{
  \position{Position1}
  \department{Department1}              %% \department is recommended
  \institution{Institution1}            %% \institution is required
  \streetaddress{Street1 Address1}
  \city{City1}
  \state{State1}
  \postcode{Post-Code1}
  \country{Country1}
}
\email{first1.last1@inst1.edu}          %% \email is recommended

\author{First2 Last2}
\authornote{with author2 note}          %% \authornote is optional;
                                      %% can be repeated if necessary
\orcid{nnnn-nnnn-nnnn-nnnn}             %% \orcid is optional
\affiliation{
  \position{Position2a}
  \department{Department2a}             %% \department is recommended
  \institution{Institution2a}           %% \institution is required
  \streetaddress{Street2a Address2a}
  \city{City2a}
  \state{State2a}
  \postcode{Post-Code2a}
  \country{Country2a}
}
\email{first2.last2@inst2a.com}         %% \email is recommended
\affiliation{
  \position{Position2b}
  \department{Department2b}             %% \department is recommended
  \institution{Institution2b}           %% \institution is required
  \streetaddress{Street3b Address2b}
  \city{City2b}
  \state{State2b}
  \postcode{Post-Code2b}
  \country{Country2b}
}
\email{first2.last2@inst2b.org}         %% \email is recommended

\begin{abstract}
% An abstract should consist of six main sentences:
%  1. Introduction. In one sentence, what’s the topic?
%  2. State the problem you tackle.
%  3. Summarize (in one sentence) why nobody else has adequately answered the research question yet.
%  4. Explain, in one sentence, how you tackled the research question.
%  5. In one sentence, how did you go about doing the research that follows from your big idea.
%  6. As a single sentence, what’s the key impact of your research?

% (http://www.easterbrook.ca/steve/2010/01/how-to-write-a-scientific-abstract-in-six-easy-steps/)

\end{abstract}

% Only add ACM notes and keywords in camera ready version
% Drop citations and footnotes in draft and blind mode.
\ifx\acmversion\acmversionanonymous
\settopmatter{printacmref=false} % Removes citation information below abstract
\renewcommand\footnotetextcopyrightpermission[1]{} % removes footnote with conference information in first column
\fi
\ifx\acmversion\acmversionjournal
%% 2012 ACM Computing Classification System (CSS) concepts
%% Generate at 'http://dl.acm.org/ccs/ccs.cfm'.
\begin{CCSXML}
<ccs2012>
<concept>
<concept_id>10011007.10011006.10011008</concept_id>
<concept_desc>Software and its engineering~General programming languages</concept_desc>
<concept_significance>500</concept_significance>
</concept>
<concept>
<concept_id>10003456.10003457.10003521.10003525</concept_id>
<concept_desc>Social and professional topics~History of programming languages</concept_desc>
<concept_significance>300</concept_significance>
</concept>
</ccs2012>
\end{CCSXML}

\ccsdesc[500]{Software and its engineering~General programming languages}
\ccsdesc[300]{Social and professional topics~History of programming languages}
%% End of generated code

%% Keywords
%% comma separated list
\keywords{keyword1, keyword2, keyword3}
\fi

%% \maketitle
%% Note: \maketitle command must come after title commands, author
%% commands, abstract environment, Computing Classification System
%% environment and commands, and keywords command.
\maketitle

% % generated by Ott 0.32 from: lc.ott
\newcommand{\ottdrule}[4][]{{\displaystyle\frac{\begin{array}{l}#2\end{array}}{#3}\quad\ottdrulename{#4}}}
\newcommand{\ottusedrule}[1]{\[#1\]}
\newcommand{\ottpremise}[1]{ #1 \\}
\newenvironment{ottdefnblock}[3][]{ \framebox{\mbox{#2}} \quad #3 \\[0pt]}{}
\newenvironment{ottfundefnblock}[3][]{ \framebox{\mbox{#2}} \quad #3 \\[0pt]\begin{displaymath}\begin{array}{l}}{\end{array}\end{displaymath}}
\newcommand{\ottfunclause}[2]{ #1 \equiv #2 \\}
\newcommand{\ottnt}[1]{\mathit{#1}}
\newcommand{\ottmv}[1]{\mathit{#1}}
\newcommand{\ottkw}[1]{\mathbf{#1}}
\newcommand{\ottsym}[1]{#1}
\newcommand{\ottcom}[1]{\text{#1}}
\newcommand{\ottdrulename}[1]{\textsc{#1}}
\newcommand{\ottcomplu}[5]{\overline{#1}^{\,#2\in #3 #4 #5}}
\newcommand{\ottcompu}[3]{\overline{#1}^{\,#2<#3}}
\newcommand{\ottcomp}[2]{\overline{#1}^{\,#2}}
\newcommand{\ottgrammartabular}[1]{\begin{supertabular}{llcllllll}#1\end{supertabular}}
\newcommand{\ottmetavartabular}[1]{\begin{supertabular}{ll}#1\end{supertabular}}
\newcommand{\ottrulehead}[3]{$#1$ & & $#2$ & & & \multicolumn{2}{l}{#3}}
\newcommand{\ottprodline}[6]{& & $#1$ & $#2$ & $#3 #4$ & $#5$ & $#6$}
\newcommand{\ottfirstprodline}[6]{\ottprodline{#1}{#2}{#3}{#4}{#5}{#6}}
\newcommand{\ottlongprodline}[2]{& & $#1$ & \multicolumn{4}{l}{$#2$}}
\newcommand{\ottfirstlongprodline}[2]{\ottlongprodline{#1}{#2}}
\newcommand{\ottbindspecprodline}[6]{\ottprodline{#1}{#2}{#3}{#4}{#5}{#6}}
\newcommand{\ottprodnewline}{\\}
\newcommand{\ottinterrule}{\\[5.0mm]}
\newcommand{\ottafterlastrule}{\\}
\newcommand{\ottmetavars}{
\ottmetavartabular{
 $ \mathit{termvar} ,\, \mathit{x} $ &  \\
 $ \ottmv{index} ,\, \ottmv{i} ,\, \ottmv{j} ,\, \ottmv{n} ,\, \ottmv{m} $ &  \\
}}

\newcommand{\ottt}{
\ottrulehead{\ottnt{t}}{::=}{\ottcom{term}}\ottprodnewline
\ottfirstprodline{|}{\mathit{x}}{}{}{}{\ottcom{variable}}\ottprodnewline
\ottprodline{|}{\lambda  \mathit{x}  \ottsym{.}  \ottnt{t}}{}{}{}{\ottcom{lambda}}\ottprodnewline
\ottprodline{|}{\ottnt{t} \, \ottnt{t'}}{}{}{}{\ottcom{application}}\ottprodnewline
\ottprodline{|}{\ottsym{(}  \ottnt{t}  \ottsym{)}} {\textsf{S}}{}{}{}\ottprodnewline
\ottprodline{|}{\ottsym{\{}  \ottnt{t}  \ottsym{/}  \mathit{x}  \ottsym{\}}  \ottnt{t'}} {\textsf{M}}{}{}{}}

\newcommand{\ottterminals}{
\ottrulehead{\ottnt{terminals}}{::=}{}\ottprodnewline
\ottfirstprodline{|}{ \lambda }{}{}{}{}\ottprodnewline
\ottprodline{|}{ \longrightarrow }{}{}{}{}}

\newcommand{\ottv}{
\ottrulehead{\ottnt{v}}{::=}{}\ottprodnewline
\ottfirstprodline{|}{\lambda  \mathit{x}  \ottsym{.}  \ottnt{t}}{}{}{}{}}

\newcommand{\ottformula}{
\ottrulehead{\ottnt{formula}}{::=}{}\ottprodnewline
\ottfirstprodline{|}{\ottnt{judgement}}{}{}{}{}}

\newcommand{\ottJop}{
\ottrulehead{\ottnt{Jop}}{::=}{}\ottprodnewline
\ottfirstprodline{|}{\ottnt{t_{{\mathrm{1}}}}  \longrightarrow  \ottnt{t_{{\mathrm{2}}}}}{}{}{}{\ottcom{$\ottnt{t_{{\mathrm{1}}}}$ reduces to $\ottnt{t_{{\mathrm{2}}}}$}}}

\newcommand{\ottjudgement}{
\ottrulehead{\ottnt{judgement}}{::=}{}\ottprodnewline
\ottfirstprodline{|}{\ottnt{Jop}}{}{}{}{}}

\newcommand{\ottuserXXsyntax}{
\ottrulehead{\ottnt{user\_syntax}}{::=}{}\ottprodnewline
\ottfirstprodline{|}{\mathit{termvar}}{}{}{}{}\ottprodnewline
\ottprodline{|}{\ottmv{index}}{}{}{}{}\ottprodnewline
\ottprodline{|}{\ottnt{t}}{}{}{}{}\ottprodnewline
\ottprodline{|}{\ottnt{terminals}}{}{}{}{}\ottprodnewline
\ottprodline{|}{\ottnt{v}}{}{}{}{}}

\newcommand{\ottgrammar}{\ottgrammartabular{
\ottt\ottinterrule
\ottterminals\ottinterrule
\ottv\ottinterrule
\ottformula\ottinterrule
\ottJop\ottinterrule
\ottjudgement\ottinterrule
\ottuserXXsyntax\ottafterlastrule
}}

% defnss
% defns Jop
%% defn reduce
\newcommand{\ottdruleaxXXapp}[1]{\ottdrule[#1]{%
}{
\ottsym{(}  \lambda  \mathit{x}  \ottsym{.}  \ottnt{t_{{\mathrm{12}}}}  \ottsym{)} \, \ottnt{v_{{\mathrm{2}}}}  \longrightarrow  \ottsym{\{}  \ottnt{v_{{\mathrm{2}}}}  \ottsym{/}  \mathit{x}  \ottsym{\}}  \ottnt{t_{{\mathrm{12}}}}}{%
{\ottdrulename{ax\_app}}{}%
}}


\newcommand{\ottdrulectxXXappXXfun}[1]{\ottdrule[#1]{%
\ottpremise{\ottnt{t_{{\mathrm{1}}}}  \longrightarrow  \ottnt{t'_{{\mathrm{1}}}}}%
}{
\ottnt{t_{{\mathrm{1}}}} \, \ottnt{t}  \longrightarrow  \ottnt{t'_{{\mathrm{1}}}} \, \ottnt{t}}{%
{\ottdrulename{ctx\_app\_fun}}{}%
}}


\newcommand{\ottdrulectxXXappXXarg}[1]{\ottdrule[#1]{%
\ottpremise{\ottnt{t_{{\mathrm{1}}}}  \longrightarrow  \ottnt{t'_{{\mathrm{1}}}}}%
}{
\ottnt{v} \, \ottnt{t_{{\mathrm{1}}}}  \longrightarrow  \ottnt{v} \, \ottnt{t'_{{\mathrm{1}}}}}{%
{\ottdrulename{ctx\_app\_arg}}{}%
}}

\newcommand{\ottdefnreduce}[1]{\begin{ottdefnblock}[#1]{$\ottnt{t_{{\mathrm{1}}}}  \longrightarrow  \ottnt{t_{{\mathrm{2}}}}$}{\ottcom{$\ottnt{t_{{\mathrm{1}}}}$ reduces to $\ottnt{t_{{\mathrm{2}}}}$}}
\ottusedrule{\ottdruleaxXXapp{}}
\ottusedrule{\ottdrulectxXXappXXfun{}}
\ottusedrule{\ottdrulectxXXappXXarg{}}
\end{ottdefnblock}}


\newcommand{\ottdefnsJop}{
\ottdefnreduce{}}

\newcommand{\ottdefnss}{
\ottdefnsJop
}

\newcommand{\ottall}{\ottmetavars\\[0pt]
\ottgrammar\\[5.0mm]
\ottdefnss}



\section{Introduction}


Our contributions are:
\begin{itemize}
\item A new tactic, \texttt{matchgoal.lean} to automate syntax-heavy proofs that occur
        commonly in programming language theory.
\item A formal semantics of the tactic, which supplements the current
        of the informal english prose of the Coq documentation.
\item A discussion of how Lean's extensible macro system makes syntactic matching
        more complex, and how we design our tactic to deal with the complexities that arise
        from Lean's extensible design.
\end{itemize}

\section{Syntax and Expr}
% https://q.uiver.app/?q=WzAsNixbMCwyLCJcXHRleHR0dHtpZiA/YiB0aGVuIDEgZWxzZSAwfSJdLFsyLDIsIlxcdGV4dHR0e2l0ZSBtdmFyLmIgKH0gXFxsYW1iZGEuIFxcdGV4dHR0e1xcX30gXFxSaWdodGFycm93IFxcdGV4dHR0ezEpfSBcXHRleHR0dHsofSBcXGxhbWJkYS4gXFx0ZXh0dHR7XFxffSBcXFJpZ2h0YXJyb3cgXFx0ZXh0dHR7MCl9Il0sWzIsMSwiXFx0ZXh0dHR7aXRlIGMgKH0gXFxsYW1iZGEuIFxcdGV4dHR0e1xcX30gXFxSaWdodGFycm93IFxcdGV4dHR0ezEpfSBcXHRleHR0dHsofSBcXGxhbWJkYS4gXFx0ZXh0dHR7XFxffSBcXFJpZ2h0YXJyb3cgXFx0ZXh0dHR7MCl9Il0sWzAsMSwiXFx0ZXh0dHR7aWYgYyB0aGVuIDEgZWxzZSAwfSJdLFsyLDAsIlxcdGV4dHR0e0V4cHJ9Il0sWzAsMCwiXFx0ZXh0dHR7U3ludGF4fSJdLFszLDAsIlxcdGV4dHR0e1VuaWZ5fSIsMix7InN0eWxlIjp7ImJvZHkiOnsibmFtZSI6InNxdWlnZ2x5In0sImhlYWQiOnsibmFtZSI6Im5vbmUifX19XSxbMiwxLCJcXHRleHR0dHtVbmlmeSB1cHRvIERlZkVxfSIsMCx7InN0eWxlIjp7ImJvZHkiOnsibmFtZSI6InNxdWlnZ2x5In0sImhlYWQiOnsibmFtZSI6Im5vbmUifX19XSxbMCwxLCJcXHRleHR0dHtlbGFifSIsMV0sWzIsMywiXFx0ZXh0dHR7ZGVsYWJ9IiwxXV0=
\[\begin{tikzcd}
	{\texttt{Syntax}} && {\texttt{Expr}} \\
	{\texttt{if c then 1 else 0}} && {\texttt{ite c (} \lambda. \texttt{\_} \Rightarrow \texttt{1)} \texttt{(} \lambda. \texttt{\_} \Rightarrow \texttt{0)}} \\
	{\texttt{if ?b then 1 else 0}} && {\texttt{ite mvar.b (} \lambda. \texttt{\_} \Rightarrow \texttt{1)} \texttt{(} \lambda. \texttt{\_} \Rightarrow \texttt{0)}}
	\arrow["{\texttt{Unify}}"', squiggly, no head, from=2-1, to=3-1]
	\arrow["{\texttt{Unify upto DefEq}}", squiggly, no head, from=2-3, to=3-3]
	\arrow["{\texttt{elab}}"{description}, from=3-1, to=3-3]
	\arrow["{\texttt{delab}}"{description}, from=2-3, to=2-1]
\end{tikzcd}\]

\section{Pure elaboration based approach}

If we start with a pattern \texttt{patStx}, elaborate it to \texttt{patExpr} so we can call
\texttt{isDefEq} on \texttt{groundExpr}, then the elaboratoin fails, because in the case
of a pattern like \texttt{if \#b then 1 else 0}, lean tries to synthesize a \texttt{DecidableEq ?b} instance
for \texttt{\#b}, which it indeed does not have enough information to create.

\section{Pure delaboration based approach}

If we start with the ground expression \texttt{grndExpr}, and we try to delaborate it into
\texttt{grndStx} to match against \texttt{patStx}, we get a strange issue where trivial
syntactic niceties like whitespace start to matter. If we try to match against \texttt{ \#x - \#x = 0},
then we get errors where \texttt{FAILURE atom '-' /= ' - ' }.

%% Acknowledgments
\begin{acks}                            %% acks environment is optional
                                        %% contents suppressed with 'anonymous'
  %% Commands \grantsponsor{<sponsorID>}{<name>}{<url>} and
  %% \grantnum[<url>]{<sponsorID>}{<number>} should be used to
  %% acknowledge financial support and will be used by metadata
  %% extraction tools.
  This material is based upon work supported by the
  \grantsponsor{GS100000001}{National Science
    Foundation}{http://dx.doi.org/10.13039/100000001} under Grant
  No.~\grantnum{GS100000001}{nnnnnnn} and Grant
  No.~\grantnum{GS100000001}{mmmmmmm}.  Any opinions, findings, and
  conclusions or recommendations expressed in this material are those
  of the author and do not necessarily reflect the views of the
  National Science Foundation.
\end{acks}

%% Bibliography
\bibliography{references}



\end{document}


%%% Local Variables:
%%% TeX-command-extra-options: "-shell-escape"
%%% End:
